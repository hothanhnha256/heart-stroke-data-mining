\section{Khảo sát và phân tích dữ liệu - EDA}

\subsection{Tổng quan về dataset}

\subsubsection{Đặc điểm cơ bản}
Dataset \textbf{Healthcare Stroke Data} được sử dụng trong dự án bao gồm thông tin y tế và nhân khẩu học của 5,110 bệnh nhân. Mục tiêu là dự đoán nguy cơ đột quỵ dựa trên 11 thuộc tính đầu vào.

\subsubsection{Cấu trúc thuộc tính}
Dataset bao gồm các loại thuộc tính sau:

\textbf{1. Biến số liên tục (Numeric):}
\begin{itemize}
    \item \texttt{age}: Tuổi (0.08 - 82)
    \item \texttt{avg\_glucose\_level}: Mức glucose trung bình (55.12 - 271.74 mg/dL)
    \item \texttt{bmi}: Chỉ số khối cơ  thể (10.3 - 97.6)
\end{itemize}

\textbf{2. Biến nhị phân (Binary):}
\begin{itemize}
    \item \texttt{hypertension}: Tăng huyết áp (0: Không, 1: Có)
    \item \texttt{heart\_disease}: Bệnh tim (0: Không, 1: Có)
\end{itemize}

\textbf{3. Biến phân loại (Categorical):}
\begin{itemize}
    \item \texttt{gender}: Giới tính (Male, Female, Other)
    \item \texttt{ever\_married}: Tình trạng hôn nhân (Yes, No)
    \item \texttt{work\_type}: Loại công việc (Private, Self-employed, Govt\_job, children, Never\_worked)
    \item \texttt{Residence\_type}: Nơi cư trú (Urban, Rural)
    \item \texttt{smoking\_status}: Tình trạng hút thuốc (formerly smoked, never smoked, smokes, Unknown)
\end{itemize}

\textbf{4. Biến mục tiêu (Target):}
\begin{itemize}
    \item \texttt{stroke}: Đột quỵ (0: Không, 1: Có)
\end{itemize}

\subsection{Phân tích biến mục tiêu (Target Analysis)}

\subsubsection{Phân phối class}
Một trong những vấn đề quan trọng nhất của dataset là \textbf{class imbalance nghiêm trọng}. Phân tích cho thấy:

\begin{equation}
P(\text{stroke} = 1) = \frac{249}{5110} = 0.0487 \approx 4.87\%
\end{equation}

\begin{equation}
P(\text{stroke} = 0) = \frac{4861}{5110} = 0.9513 \approx 95.13\%
\end{equation}

\begin{equation}
\text{Imbalance Ratio} = \frac{N_{\text{majority}}}{N_{\text{minority}}} = \frac{4861}{249} \approx 19.5
\end{equation}

Sự mất cân bằng này có ý nghĩa quan trọng:
\begin{itemize}
    \item Models có xu hướng bias về lớp đa số (No Stroke)
    \item Accuracy không phải là metric đánh giá phù hợp
    \item Cần áp dụng techniques như SMOTE để cân bằng dữ liệu training
    \item Metrics như F1-Score, Recall, ROC-AUC quan trọng hơn
\end{itemize}

\textbf{Visualization}: Hình \ref{fig:target_distribution} minh họa phân phối không cân bằng của biến target.

\begin{figure}[H]
    \centering
    \includegraphics[width=1\linewidth]{image//1.5/target_stroke.png}
    \vspace{0.5em}
    \caption{Phân phối biến target - Class Imbalance nghiêm trọng (95.13\% vs 4.87\%)}
    \label{fig:target_distribution}
\end{figure}

\subsection{Phân tích biến số (Numeric Analysis)}

\subsubsection{Thống kê mô tả}
Ba biến số chính trong dataset được phân tích với các thống kê mô tả:

\begin{table}[h]
\centering
\caption{Thống kê mô tả các biến số}
\begin{tabular}{lrrr}
\hline
\textbf{Metric} & \textbf{age} & \textbf{avg\_glucose\_level} & \textbf{bmi} \\
\hline
Count & 5,110 & 5,110 & 4,909 \\
Mean & 43.23 & 106.15 & 28.89 \\
Std Dev & 22.61 & 45.28 & 7.85 \\
Min & 0.08 & 55.12 & 10.30 \\
25\% & 25.00 & 77.24 & 23.50 \\
Median & 45.00 & 91.88 & 28.10 \\
75\% & 61.00 & 114.09 & 33.10 \\
Max & 82.00 & 271.74 & 97.60 \\
\hline
\end{tabular}
\label{tab:numeric_stats}
\end{table}

\subsubsection{Phát hiện outliers}
Phân tích box plots và histograms cho thấy:

\textbf{1. Age (Tuổi):}
\begin{itemize}
    \item Phân phối tương đối đều, có một số trường hợp trẻ em (< 1 tuổi)
    \item Không có outliers nghiêm trọng
    \item Xu hướng: Nguy cơ đột quỵ tăng theo tuổi
\end{itemize}

\textbf{2. Average Glucose Level:}
\begin{itemize}
    \item Phân phối lệch phải (right-skewed)
    \item Có outliers với giá trị cao (> 200 mg/dL)
    \item Cần áp dụng IQR capping trong preprocessing
\end{itemize}

\textbf{3. BMI:}
\begin{itemize}
    \item Phân phối gần chuẩn (near-normal)
    \item Có outliers ở cả hai đầu (< 15 hoặc > 50)
    \item Missing values: 201 mẫu (3.93\%) - cần imputation
\end{itemize}

\begin{figure}[h]
\centering
% \includegraphics[width=0.95\textwidth]{eda/eda_numeric_analysis.png}
\caption{Phân tích biến số: Histograms và Box plots theo stroke status}
\label{fig:numeric_analysis}
\end{figure}

\subsubsection{Correlation với target}
Phân tích correlation Pearson với biến target cho thấy:

\begin{table}[h]
\centering
\caption{Correlation của biến số với stroke}
\begin{tabular}{lc}
\hline
\textbf{Feature} & \textbf{Pearson Correlation (absolute)} \\
\hline
age & 0.2453 \\
avg\_glucose\_level & 0.1319 \\
bmi & 0.0361 \\
\hline
\end{tabular}
\label{tab:numeric_correlation}
\end{table}

Nhận xét:
\begin{itemize}
    \item \textbf{age} có correlation mạnh nhất với stroke (0.245)
    \item \textbf{avg\_glucose\_level} có correlation trung bình (0.132)
    \item \textbf{bmi} có correlation yếu (0.036)
\end{itemize}

\subsection{Phân tích biến phân loại (Categorical Analysis)}

\subsubsection{Phân phối và stroke rate theo từng biến}

\textbf{1. Gender (Giới tính):}
\begin{itemize}
    \item Female: 2,994 mẫu (58.6\%)
    \item Male: 2,115 mẫu (41.4\%)
    \item Other: 1 mẫu (0.02\%)
    \item Stroke rate: Female (4.71\%), Male (5.11\%), Other (0\%)
\end{itemize}

\textbf{2. Ever Married (Tình trạng hôn nhân):}
\begin{itemize}
    \item Yes: 3,353 mẫu (65.6\%)
    \item No: 1,757 mẫu (34.4\%)
    \item \textbf{Quan sát quan trọng}: Người đã kết hôn có stroke rate cao hơn (6.5\% vs 1.3\%)
    \item Có thể do correlation với tuổi (người đã kết hôn thường lớn tuổi hơn)
\end{itemize}

\textbf{3. Work Type (Loại công việc):}
\begin{itemize}
    \item Private: 2,925 mẫu (57.2\%)
    \item Self-employed: 819 mẫu (16.0\%)
    \item children: 687 mẫu (13.4\%)
    \item Govt\_job: 657 mẫu (12.9\%)
    \item Never\_worked: 22 mẫu (0.4\%)
    \item Stroke rate cao nhất: Self-employed (7.94\%), thấp nhất: children (0.29\%)
\end{itemize}

\textbf{4. Residence Type (Nơi cư trú):}
\begin{itemize}
    \item Urban: 2,596 mẫu (50.8\%)
    \item Rural: 2,514 mẫu (49.2\%)
    \item Stroke rate: Urban (5.20\%) vs Rural (4.53\%) - Sự khác biệt không lớn
\end{itemize}

\textbf{5. Smoking Status (Tình trạng hút thuốc):}
\begin{itemize}
    \item never smoked: 1,892 mẫu (37.0\%)
    \item Unknown: 1,544 mẫu (30.2\%)
    \item formerly smoked: 885 mẫu (17.3\%)
    \item smokes: 789 mẫu (15.4\%)
    \item Stroke rate cao nhất: formerly smoked (7.91\%), Unknown có rate thấp nhất (3.04\%)
\end{itemize}

\begin{figure}[h]
\centering
% \includegraphics[width=0.95\textwidth]{eda/eda_categorical_analysis.png}
\caption{Phân tích biến phân loại: Count plots phân biệt theo stroke status}
\label{fig:categorical_analysis}
\end{figure}

\subsection{Phân tích tương quan (Correlation Analysis)}
\begin{figure}[H]
    \centering
    \includegraphics[width=1\linewidth]{image//1.5/correlation_matrix.png}
    \caption{Enter Caption}
    \label{fig:placeholder}
\end{figure}
\subsubsection{Ma trận correlation}
Để phân tích correlation giữa các biến, các biến categorical được encode thành numeric codes. Ma trận correlation hoàn chỉnh cho thấy:

\begin{table}[h]
\centering
\caption{Top correlations với biến target stroke}
\begin{tabular}{lc}
\hline
\textbf{Feature} & \textbf{Correlation (absolute)} \\
\hline
age & 0.2453 \\
heart\_disease & 0.1349 \\
avg\_glucose\_level & 0.1319 \\
hypertension & 0.1279 \\
ever\_married & 0.1083 \\
bmi & 0.0361 \\
work\_type & 0.0323 \\
smoking\_status & 0.0281 \\
Residence\_type & 0.0155 \\
gender & 0.0089 \\
\hline
\end{tabular}
\label{tab:all_correlation}
\end{table}

\textbf{Insights quan trọng:}
\begin{itemize}
    \item \textbf{Age} là predictor mạnh nhất (r = 0.245)
    \item \textbf{Heart disease} và \textbf{Hypertension} có correlation trung bình (0.13-0.14)
    \item \textbf{Residence\_type} và \textbf{Gender} có correlation rất yếu (< 0.02)
    \item Không có multicollinearity nghiêm trọng giữa các features
\end{itemize}

\begin{figure}[h]
\centering
% \includegraphics[width=0.9\textwidth]{eda/eda_correlation_matrix.png}
\caption{Ma trận correlation heatmap - Phân tích mối quan hệ giữa các biến}
\label{fig:correlation_matrix}
\end{figure}

\subsection{Phân tích chi tiết biến Age}

Do \textbf{age} là biến có correlation cao nhất với stroke, một phân tích chuyên sâu được thực hiện:

\subsubsection{Phân chia nhóm tuổi}
Dữ liệu được chia thành 4 nhóm tuổi:
\begin{itemize}
    \item \textbf{< 30}: Trẻ và người trẻ tuổi
    \item \textbf{30-50}: Trung niên
    \item \textbf{50-65}: Người lớn tuổi
    \item \textbf{65+}: Người cao tuổi
    
\end{itemize}
\begin{figure}[H]
    \centering
    \includegraphics[width=1\linewidth]{image//1.5/age.png}
    \caption{Enter Caption}
    \label{fig:placeholder}
\end{figure}

\subsubsection{Phát hiện chính}
\begin{enumerate}
    \item \textbf{Xu hướng tăng mạnh}: Tỷ lệ stroke tăng gấp 124 lần từ nhóm < 30 (0.13\%) đến nhóm 65+ (16.17\%)
    
    \item \textbf{Ngưỡng quan trọng}: Nguy cơ tăng đáng kể sau 50 tuổi
    \begin{equation}
    \text{Risk Ratio}_{65+/<30} = \frac{16.17\%}{0.13\%} \approx 124
    \end{equation}
    
    \item \textbf{Phân phối không đồng đều}: Nhóm < 30 chiếm 30.7\% mẫu nhưng chỉ có 0.8\% ca stroke
    
    \item \textbf{Nhóm cao rủi ro}: Người cao tuổi (65+) chiếm 18.9\% mẫu nhưng có 62.7\% tổng số ca stroke
\end{enumerate}

\begin{figure}[h]
\centering
% \includegraphics[width=0.95\textwidth]{eda/eda_age_analysis.png}
\caption{Phân tích chi tiết tuổi: Phân phối và tỷ lệ stroke theo nhóm tuổi}
\label{fig:age_analysis}
\end{figure}

\subsection{Phát hiện và kết luận từ EDA}

\subsubsection{Phát hiện chính}
\begin{enumerate}
    \item \textbf{Class Imbalance nghiêm trọng}: 
    \begin{itemize}
        \item Tỷ lệ 19.5:1 (4,861 vs 249)
        \item Cần SMOTE hoặc class\_weight trong training
        \item Metrics: Ưu tiên F1-Score, Recall, ROC-AUC thay vì Accuracy
    \end{itemize}
    
    \item \textbf{Biến quan trọng nhất - Age}:
    \begin{itemize}
        \item Correlation cao nhất (0.245)
        \item Tỷ lệ stroke tăng 124 lần từ < 30 đến 65+
        \item Có thể áp dụng feature engineering (age groups, polynomial features)
    \end{itemize}
    
    \item \textbf{Missing Values}:
    \begin{itemize}
        \item Chỉ BMI có 201 missing values (3.93\%)
        \item Cần imputation strategy: median/mean/KNN imputation
    \end{itemize}
    
    \item \textbf{Outliers}:
    \begin{itemize}
        \item avg\_glucose\_level: Nhiều giá trị cao > 200
        \item BMI: Một số giá trị cực đoan (< 15 hoặc > 50)
        \item Áp dụng IQR capping trong preprocessing
    \end{itemize}
    
    \item \textbf{Categorical Variables}:
    \begin{itemize}
        \item ever\_married có correlation với age và stroke (Yes: 6.56\% vs No: 1.65\%)
        \item smoking\_status: "formerly smoked" có stroke rate cao (7.91\%)
        \item Residence\_type: Urban (5.20\%) vs Rural (4.53\%) - khác biệt nhỏ
        \item Gender: Female (4.71\%) vs Male (5.11\%) - tương đương nhau
    \end{itemize}
    
    \item \textbf{Feature Engineering Opportunities}:
    \begin{itemize}
        \item Age groups (binning)
        \item Interaction features: age × hypertension, age × heart\_disease
        \item Polynomial features cho age
    \end{itemize}
\end{enumerate}

\subsubsection{Hướng tiếp cận preprocessing}
Dựa trên EDA, pipeline preprocessing cần bao gồm:

\begin{enumerate}
    \item \textbf{Missing Value Handling}:
    \begin{itemize}
        \item BMI: SimpleImputer với strategy='median'
    \end{itemize}
    
    \item \textbf{Outlier Treatment}:
    \begin{itemize}
        \item IQR capping cho avg\_glucose\_level và BMI
        \item Công thức: $[\text{Q1} - 1.5 \times \text{IQR}, \text{Q3} + 1.5 \times \text{IQR}]$
    \end{itemize}
    
    \item \textbf{Scaling}:
    \begin{itemize}
        \item StandardScaler cho numeric features (age, avg\_glucose\_level, BMI)
        \item Quan trọng cho SVM, KNN, Logistic Regression
    \end{itemize}
    
    \item \textbf{Encoding}:
    \begin{itemize}
        \item OneHotEncoder cho categorical variables
        \item handle\_unknown='ignore' để xử lý unseen categories
    \end{itemize}
    
    \item \textbf{Balancing}:
    \begin{itemize}
        \item SMOTE oversampling cho training set
        \item Giữ nguyên test set distribution để đánh giá realistic
    \end{itemize}
\end{enumerate}

\subsubsection{Kết luận}
EDA cho thấy dataset có những đặc điểm:
\begin{itemize}
    \item \textbf{Strengths}: Dữ liệu sạch với ít missing values, các features có ý nghĩa y tế rõ ràng
    \item \textbf{Challenges}: Class imbalance nghiêm trọng, outliers trong biến số, correlation yếu giữa hầu hết features với target
    \item \textbf{Key Insight}: Age là predictor mạnh nhất, cần đặc biệt chú ý trong modeling
    \item \textbf{Next Steps}: Áp dụng preprocessing pipeline toàn diện với SMOTE, scaling, encoding
\end{itemize}

Các visualizations từ EDA được sử dụng để hiểu sâu về dữ liệu và định hướng các bước tiền xử lý, feature selection, và model training tiếp theo.
